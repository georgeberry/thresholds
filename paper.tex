\documentclass{article}
\usepackage[utf8]{inputenc}

\title{threshold-model}
\author{George Berry}
\date{October 2014}

\begin{document}

\maketitle

\section{Introduction}

\section{Theoretically Defining a Threshold}

$\theta_v$ is the minimum number of alters that need to adopt in order to influence ego to adopt. Simiarly, but not equivalently, $\theta_v$ may be defined as the fraction of alters needed to influence ego to adopt. As Macy and Centola (2007) point out, the fractional approach implicitly weights the presence of non-adopters. The fractional approach has been used to model direct-benefit variants of threshold models, where the network externalities of a product become strong when a large fraction alters use the product. A more traditional integer-valued threshold is often more appropriate in sitautions where information facilitates diffusion. 

Both the fractional and integer-valued approach will be addressed here, but for simplicity I begin with an integer-valued threshold, much like Granovetter (1978). The literature often enforces the assumption of a uniform distribution of thresholds (Granovetter (1978), Kempe et al. (2003), Kempe et al. (2005), Goyal (2010)), but I enforce no such restriction at this point. $\theta_v$ is simply $min(|S|): y_v = 1$, or the minimum size of the set of $v$'s alters that have adopted, $S$.

This formulation assumes that in the absence of social information, ego would not adopt. In the case where ego would adopt with no social information, $\theta_v = 0$ and we refer to $v$ as an early adopter (also known as an entrepreneur or seed node).

Questions of where $\theta_v$ comes from have not been well-addressed in the literature. In Granovetter (1978), thresholds are given in advance by the modeler in search of a theoretical goal. For Granovetter, his choice of actor thresholds allowed him to cleanly demonstrate a that observed outcomes are not necessarily strongly correlated with individuals' willingness to contribute to those outcomes. Similarly, Schelling (1978) freely manipulates thresholds to demonstrate interesting properties of social dilemmas when various distributions take hold. Kempe et al. (2003) and Watts (2002) both draw thresholds from a uniform distribution, a common choice when theoretically and formally examining this genus of model.

Thresholds, then, are often freely manipulated by the researcher for her own modeling purposes. This leaves something to be desired, espeically in light of the well-estabalished importance of even minor deviations in the threshold distribution to social outcomes. For instance, the most common threshold distribution is unform---yet I am aware of no studies that try to empirically investigate this claim that normally distributed thresholds are mainstays of human society across cultures and innovations.

I propose a somewhat different take on thresholds: a threshold is, in the social context of the moment of adoption, the minimal number of alters ego needs to partake in a social action, begin to use an innovation, or begin to adhere to a belief. This formulation does not change any of the well-established formal results surrounding thresholds, but shifts emphasis to the fact that a threshold is generated by a set of individual, environmental, and structural factors. 

It does, however, clarify the importance---and emphemerality---of thresholds. Call an operationalized measure of an expansive notionn of social context contained in an arbitrary number of variables $X$. It is plain that $\theta_v$ depends on $X$: small manipulations in social context have been shown to produce large behavioral changes <<Ziggy paper from lab in Science>>. The idea that social context affects thresholds is largely absent from the literature: thresholds are considered fixed in an almost metaphysical sense. However, consider a thought experiment using Granovetter's riot model: in one ``state of the world'', it is pouring rain; in the other, it is sunny. In the former state, even ardent rioters may need more of a social incentive to start rioting; in the latter, fair weathered rioters will need fewer fellow rioters to riot (because it is fair weathered).

I therefore propose that $\theta_v$ is best understood as outcome, rather than a hidden quantity used as a modeling convenience. A threshold is an outcome of contextual, and structural characteristics. I denote these $X_i$, $X_c$, and $X_s$, with $X$ being used to denote the amalgamation of these factors. It signifies the point at which the benefit of the action is greater than the cost, or $U_v|X > C_v|X$, where $U_v$ stands for the utility of the action to $v$ and $C_v$ stands for the cost.

This raises the question why we focus on the number of alters as they contribute to thresholds rather than other characteristics, such as the weather. There are two reasons, one theoretical and one practical. As a theoretical matter, sociologists are interested in the effects that social factors have on behavior; the number of alters that have adopted something is a key social factor. As a practical matter for policy, changing the number of an individual's friends that have adopted a behavior is extremely difficult. We would like to focus on simpler manipulations that promote prosocial behavior. It makes sense to focus on easier-to-manipulate interventions that can potentially lower thresholds, as a way to examine the possibility of starting a sustainable cascade of desired behavior. 

This practical element is the difference between:
$\theta_v = f(p)$ and $p = f(\theta_v)$, where $p$ is the presence of absence of a policy intervention.


//THIS NEEDS TO BE REDONE
Following Pearl (2000), one of the main benefits of well-formed models is the ability to use them to construct policy recommendations. This relies on modularity of the intervention, meaning it can be turned on and off at will, without changing other factors in a causal graph. It is unlikely that we can practicably manipulate $\theta_v$, the number of alters adopting a behavior, even if $\theta_v$ were conceived of as exogenous and modular. Further, $\theta_v$ is clearly neither exogenous nor modular, as thresholds are caused by $X$ and it is a graph theoerical fact that increaing the degree of the adopting neighborhood affects other factors with probability 1 (such as clustering in the adoption network, etc.).







\subsection{Ideas}

If thresholds are a function of the world, they can be changed

Policy interventions

Good thing to think about: how many friends have to quit smoking before you do?

Important distinction (Watts 2002; Centola and Macy 2007): absolute number or fraction of friends
How much does this matter for estimation?

Homophily of thresholds: does a clique having high thresholds imply that each person in that clique should be able to convert more high threshold friends? Intuition is no.

A really restrictive/rigorous treatment would exclude everyone with estimated $\theta = 1$ to avoid the knowing-about-the-thing confound.

\end{document}
